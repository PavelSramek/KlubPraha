\documentclass[11pt,a4paper,czech]{article}
\input{files/styles/header.tex}
\input{files/styles/pirati-klub/template.tex}

\begin{document}

\recipient[]{Hlavní město Praha \\ Mariánské náměstí 2/2 \\
110\,01 \quad Praha 1 

\bigskip
\DS{48ia97h}}

\def \yoursign { }
\def \yourdate { }
\def \oursign {ZK Pha 4/2014}
\def \place {Praha}

\printheader

\subject{Žádost zastupitele hl. m. Prahy o informace}

Vážení,

v souladu s § 51 odst. 3 zákona č. 131/2000 Sb., o hlavním městě Praze, ve znění pozdějších předpisů, Vás žádám o poskytnutí informací o výdajích hl. m. Prahy evidovaných ve Vašem účetním nebo jiném podobném informačním systému. 

Žádáme o export těchto dat (tabulek databáze včetně její struktury) do elektronické podoby včetně všech atributů jednotlivých záznamů, a to pro data od 1. 1. 2010 do dne podání této žádosti. U každého příjemce veřejných prostředků nechť jsou uvedeny aspoň základní osobní údaje podle § 8b odst. 3 zákona č. 106/1999 Sb., o svobodném přístupu k informacím, ve znění pozdějších předpisů.

Jako způsob předání navrhuji, abyste mi požadované informace nahráli na externí harddisk, který dodám, v domluveném termínu. 

Alternativně žádám o zpřístupnění rozhraní, ve kterém získám přístup ke všem uvedeným datům online s možností vyhledávání (a to případně i prostřednictvím počítače – tabletu, který mi bude jako zastupiteli přidělen).

Žádám o přímé poskytnutí informace, nikoliv jen odkaz na zveřejněný dokument, a to v elektronické podobě. Předem děkuji za Vaši vstřícnost.

\signature{Mgr. Bc. Jakub Michálek \\ předseda klubu Pirátů \\ zastupitelstvo hl. m. Prahy}

%\attachments
\end{document}
